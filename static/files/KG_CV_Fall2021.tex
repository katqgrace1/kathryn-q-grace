\documentclass[11pt]{article} %Sets the default text size to 11pt and class to article.
\usepackage[letterpaper,margin=1in]{geometry}
\usepackage{hyperref}

%------------------------Dimensions--------------------------------------------
%\topmargin=0.0in %length of margin at the top of the page (1 inch added by default)
%\oddsidemargin=0.0in %length of margin on sides for odd pages
%\evensidemargin=0in %length of margin on sides for even pages
%\textwidth=6.5in %How wide you want your text to be
%\marginparwidth=0.5in
\headheight=1pt %1in margins at top and bottom (1 inch is added to this value by default)
\headsep=0pt %Increase to increase white space in between headers and the top of the page
\textheight=9.0in %How tall the text body is allowed to be on each page


\begin{document}

\centerline{\Large\textbf{Kathryn Grace}}  

\bigskip
\noindent 
Department of Geography, Environment and Society\\
558 Social Sciences Building\\
267 19th Avenue S.\\
Minneapolis, MN 55455\\
   {\tt klgrace@umn.edu} \\
\noindent
\line(1,00){470}
\bigskip

%---------------------------------------------------------------------------



\noindent
\textbf{Education}


\noindent
\begin{tabular}{lp{12cm}}
 \textbf{PhD Geography, University of California, Santa Barbara, 2008}\\
 Dissertation: ``Three Essays on Fertility and Family Planning in Guatemala"\\
\textbf {MA Statistics, University of California, Santa Barbara, 2006}\\
\textbf {MSPH Biostatistics, Tulane University School of Public Health, 2004}\\
\textbf {BA Theoretical Mathematics, University of California, Berkeley, 2002}\\
{\it Minor in French Language and Culture}\\
\end{tabular}

\vspace{.5 cm}
%---------------------------------------------------------------------------


\noindent
\textbf{Appointments}


\noindent
\begin{tabular}{lp{12cm}}
%Beginning January 2016 & Assistant Professor, Department of Geography, Environment and Society, %University of Minnesota, Twin Cities\\
2018-  & Associate Professor, Department of Geography, Environment and Society, University of Minnesota, Twin Cities\\
2016- 2018 & Assistant Professor, Department of Geography, Environment and Society, University of Minnesota, Twin Cities\\
2012-present & Research Scientist, Climate Hazards Group, University of California at Santa Barbara \\
2012-2015 & Assistant Professor, Department of Geography, University of Utah \\
2009-2012 & Postdoctoral Researcher, Climate Hazards Center, University of California at Santa Barbara \\
2008-2009 & Postdoctoral Fellow, Max Planck Institute for Demographic Research \\
\end{tabular}

\vspace{.5 cm}

\noindent
\textbf{Affiliations and Visiting Appointments}

\noindent
\begin{tabular}{lp{12cm}}
2016-present & Affiliate, Minnesota Population Center\\
2016-present& Graduate Faculty in Population Studies, Minnesota Population Center\\
Summer 2020 & Visiting Researcher, Institut National d'Etudes Demographiques, France (postponed due to Covid-19)\\
Summer 2020 & Visiting Researcher, Stockholm University Demography Unit, Sweden (postponed due to Covid-19)\\ 
Summer 2019 & Visiting Researcher, Max Planck Institute for Demographic Research, Germany\\
Summer 2018 & Visiting Researcher, Stockholm University Demography Unit, Sweden\\ 
Summer 2017 & Visiting Researcher, Stockholm University Demography Unit, Sweden\\ 
Summer 2016 & Visiting Researcher, Institut National d'Etudes Demographiques, France\\
Summer 2015 & Visiting Researcher, Institut National d'Etudes Demographiques, France\\
Summer 2013 & Visiting Researcher, Vienna Institute for Demography, Austria\\
Summer 2013 & Visiting Researcher, University of Ouagadougou, Burkina Faso\\
2012-present & Affiliate, Broom Center for Demography, University of California at Santa Barbara \\
2013-2015 & Affiliate, Latin American Studies Program, University of Utah\\
\end{tabular}

\vspace{.5 cm}

\bigskip
\noindent
\textbf{Additional Training}


\noindent
\begin{tabular}{lp{12cm}}
2014 & University of Utah Pediatric Grant Writing Workshop in Deer Valley \\
2009 & University of Rome -- Spatial Econometrics Advanced Institute \\
2009 & Max Planck Institute for Demographic Research -- Fertility Analysis and Modeling \\
2008 & University of Texas -- Dallas Summer Spatial Filtering Workshop \\
2005 & Stanford Workshop in Formal Demography \\
\end{tabular}


\vspace {.5 cm}

\noindent
\textbf {Fellowships and Honors}

\noindent
University of Minnesota College of Liberal Arts Scholar of the College, 2019-2022\\
Institut National d'Etudes Demographiques Visiting Researcher Fellowship, 2020\\
Institut National d'Etudes Demographiques Visiting Researcher Fellowship, 2015\\
Demographic and Health Survey (DHS) Population and Reproductive Health Fellowship, 2007\\
University of California Graduate Opportunity Fellowship, 2004-05 \&\ 2007-08\\
University of California Department of Geography Block Grant Funding, 2006\\
Tulane University School of Public Health Dean's Grant, 2004\\
Tulane University Public Health Traineeship Award, 2003\\           
University of California Faculty Wives Scholarship, 2001\\
University of California Education Abroad Program Scholarship, 1999\\


\noindent
\textbf {Grants and Awards} (funding amounts reflect amount received by UMN)

\begin{tabular}{lp{11cm}}

\$15,000 &\emph{The Impacts Of Extreme Weather On Older Adults’ Time Use} Life Course Center of the University of Minnesota, Pilot Grant Program\\
&PI: Sarah Flood, Jesse Berman, Kathryn Grace, Dave Van Riper\\
&Award Period:2020-2021\\

\$150,000 &\emph{Geospatial, Farming Systems, and Digital Tools Consortium – building a new era of Predictive Agricultural Innovation to improve the livelihood of smallholder farmers} Feed the Future Innovation Lab for Collaborative Research on Sustainable Intensification (SIIL), USAID\\
&PI: Ignacio Ciampitti, Kansas State University\\
&UMN co-PI: Kathryn Grace\\
& Award Period: 2020-2024\\


\$300,000 & \emph{The impact of urban and peri-urban agriculture on hydrometerology, urban food security and human health} NASA Research Opportunities in Space and Earth Sciences (subaward from \$1,300,000 grant)\\
&PI: Christopher Hain\\
&Co-Investigator: Kathryn Grace\\
&Award Period: 2020-2023\\
  

\$2,205,854 & \emph{IPUMS-PMA: Integrated Public Health Microdata Series for PMA2020} Bill \& Melinda Gates Foundation\\
&PI: Elizabeth Heger Boyle\\
&Co-Investigator: Kathryn Grace, Matt Sobek, Devon Kristainsen\\
&Award Period: 2020-2023 \\

\end{tabular}

\begin{tabular}{lp{11cm}}


\$3,105,548 & \emph{Microdata for Research on Aging in the Global South} National Institutes of Health\\
&PI: Steve Ruggles\\
&Co-Investigator: Lara Cleveland, Kathryn Grace, Matt Sobek\\
&Award Period: 2019-2024\\




\$79,857 & \emph{Connecting West Africa users to cutting edge resources: Integrating satellite observations and sub-seasonal climate forecasts to enhance agricultural and pastoral water-management decision-making using 21st century agro-pastoral water deficit predictions} NASA Research Opportunities in Space and Earth Sciences SERVIR (subaward from \$700,000 grant)\\
&PI: Shraddhanand Shukla\\
&Co-Investigator: Kathryn Grace\\
&Award Period: 2020-2023\\



\$118,000 & \emph{Modelling Early Risk Indicators to Anticipate Malnutrition - MERIAM} United Kingdom Department of International Development (DFID) (subaward from \$3,000,000 grant)\\
&PI: Kathryn Grace (Minnesota)\\
&Award Period: 2019-2021\\




\$38,671 &\emph{The Impact of Exposure to Heatwaves Across the Reproductive Life Course} UMN Talle Faculty Research Awards\\
&PI: Kathryn Grace\\
&Award Period: 2019-2021\\



\$157,141 &\emph{Early identification and forecasts of reduced-yield agricultural seasons in the Developing World} US Agency for International Development (USAID) (subaward from \$1,808,000 grant)\\
&PI: Gregory Husak (University of Calfornia, Santa Barbara)\\
&Co-Investigator: Kathryn Grace (Minnesota lead)\\
&Award Period: 2019-2024\\



\$5,047,150 & \emph{International Integrated Microdata Series} National Science Foundation\\
&PI: Steven Ruggles\\
&Co-PI: Kathryn Grace, Elizabeth Heger Boyle, Deborah Levison, Ragui Assad\\
&Award Period: 2020-2025\\



\$899,358 & \emph{Integrating and Disseminating PMA2020 Data: Towards Greater Access to Family Planning for Women in High Fertility Countries} Bill \& Melinda Gates Foundation\\
&PI: Elizabeth Heger Boyle\\
&Co-Investigator: Kathryn Grace (from 2018)\\
&Award Period: 2017-2020 \\





\$40,000& \emph{Hotspotting Cardiometabolic Disparities for Simulated Advances in Population Care} National Institute on Aging, National Institutes of Health, R01 AG052533 (subaward from \$1,500,000 grant)\\
&PI: Bjorn Westgard (University of Minnesota Medical School)\\
&Co-Investigator: Kathryn Grace (Minnesota)\\
&Award Period: 2018-2022\\



\$400,492& \emph{Understanding multi-scale resilience options for climate vulnerable Africa; Innovations at the Nexus of Food, Energy and Water Systems (INFEWS)} National Science Foundation (subaward from \$2,799,021 grant)\\
&PI: Benjamin Zaitchick (Johns Hopkins)\\
&PI: Kathryn Grace (Minnesota lead)\\
&Award Period: 2016-2020\\

\end{tabular}

\begin{tabular}{lp{11cm}}

\$4,000& \emph{A comparative analysis of the relationship between climate variability, agricultural specialization and children's health:  evidence from Kenya and Mali}; Minnesota Population Center Collaborative grant\\
&PI: Maryia Bahktsiyarava (graduate student) and co-PI: Kathryn Grace\\
&Award Period: 2016-2017\\




\$1,575& \emph{Can migrants' use of cell phones reduce the climate vulnerability of their communities of origin?}; Institute on the Environment (IonE) minigrant\\
&PI:Kathryn Grace, Jude Mikal, Jack DeWaard\\
&Award Period: 2016-2017\\






\$88,000& \emph{Estimating Micro-level Livelihood Characteristics for Vulnerability Mapping Using Household-level Surveys and Remotely Sensed Data}; Kimetrica, a subcontract through US Agency for International Development\\
&PI: Kathryn Grace\\
&Award Period: 2016\\
&note: funding reduced during project period to \$16,000 because of changes in available funds to Kimetrica\\



\$10,000& \emph{Mining Impacts: Andean Nation Development of Environmental/Economic Solutions}; University of Utah's Global Learning Across the Disciplines\\
&co-PI: Kathryn Grace\\
&Award Period: 2015\\



\$21,000& \emph{Children's health, water source and agricultural production in a context of climate change: a spatial case study of Burkina Faso}; University of Utah Primary Childrens Hospital Fellowship\\
&PI: Kathryn Grace, Mentor: Ken Smith\\
&Award Period:2015\\



\$12,000 & \emph{Central American Fertility in Los Angeles Translation: Informing Statistical Models with Qualitative Context}; University of California, Center for New Racial Studies\\
& co-PI: Kathryn Grace\\
&Award Period: 2013-2014\\

\$178,000 & \emph{Using Very High Resolution Remotely Sensed Data to Measure Cultivated Area and Land Use in High Fertility Countries}; United States Geological Service (subaward from \$1,000,000 award to UCSB Climate Hazards Group, Famine Early Warning Systems)\\
& PI: Kathryn Grace\\
& Award Period: 2012-2018\\


\$274,000 & \emph{Examining the Links Between Agriculture and Human Health in a Context of Climate Change: A Case Study of Three West African Countries - Niger, Burkina Faso and Mali}; NASA Research Opportunities in Space and Earth Sciences\\
 & PI: Kathryn Grace; Co-I: Greg Husak\\
 & Award Period: 2012-2016\\
 
 \end{tabular}

\begin{tabular}{lp{11cm}}

\$3,500 &\emph{How Food Insecurity Impacts Contraceptive Use in Early Transitional Societies}; College of Social and Behavioral Sciences, University of Utah\\
& PI: Kathryn Grace\\
& Award Period: 2012-2013\\
\end{tabular}

\vspace{.5cm}
\noindent
{\large \textbf{PUBLICATIONS}}
%\vspace{.5cm}

\noindent
\textbf {Refereed Articles (42); Refereed Book Chapters (1)}\\
\noindent
* indicates postdoctoral researcher or graduate student at the time of submission\\

\noindent 
Tuholske*, C., Caylor, K., Funk, C., Verdin, A., Sweeney, S., \textbf{Grace, K.}, Peterson, P., Evans, Tom. 2021. Global urban population exposure to extreme heat. \textit{Proceedings of the National Academy of Sciences}, In Press.\\

\noindent
Brown, M., \textbf{Grace, K.}, Billing*, T., Backer, D. 2021. Considering climate and conflict conditions together to improve interventions that prevent child acute malnutrition. \textit{The Lancet Planetary Health}. 5 (9), e654-e658.\\

\noindent
Pinchoff, J., Turner*, W., \& \textbf{Grace, K.} 2021. The association between agricultural conditions and multiple dimensions of undernutrition in children 6-23 months of age in Burkina Faso. \textit{Environmental Research Communications}, 3(6).\\

\noindent
Bakhtsiyarava*, M., \textbf{Grace, K.} 2021. Agricultural production diversity and child nutrition in Ethiopia. \textit{Food Security}, 6(5).\\

\noindent 
Bell, A. R., Roberts*, M., \textbf{Grace, K.}, Morgan*, A., Tamal, M.E.H., Killlilea, M., Ward., P. 2021. How high frequency food diaries can transform understanding of food security.  \textit{Environmental Research Letters}, 16(4).\\

\noindent
Randell, H., \textbf{K. Grace}, M. Bahktsiyarava*. 2021. Climatic Conditions and Infant Care: Implications for Child Nutrition in Rural Ethiopia. \textit{Population and Environment}, 1-29.\\

\noindent
\textbf{Grace, K.}, F. Davenport. 2021. Climate variability and health in extremely vulnerable communities: Investigating variations in surface waterhole conditions and food security in the West African Sahel. \textit{Population and Environment},  1-25.\\

\noindent
Verdin, A., \textbf{Grace, K.}, Funk, C., Husak, G., Davenport, F. 2021. Can we advance individual-level heat-health research through the application of stochastic weather generators? \textit{Climatic Change}, 164(1): 1-13.\\

\noindent
\textbf{Grace, K.}, A. Verdin, A. Dorelien, F. Davenport, G. Husak, C. Funk. 2021. Exploring strategies for investigating the mechanisms linking climate and individual-level child health outcomes: an analysis of birth weight in Mali. \textit{Demography}, 58(2): 499-526.\\

\noindent
\textbf{Grace, K.}, Siddiqui, S., Zaitchik, B. 2021. A Framework for Interdisciplinary Research In Food Systems. \textit{Nature Food}, 2(1):1-3.\\


\noindent
DiClemente*, K., \textbf{ Grace, K.}, Kershaw, T., Bosco*, E., and Humphries, D. 2020. Investigating the Relationship between Food Insecurity and Fertility Preferences in Tanzania. \textit{Maternal and Child Health Journal}, 25(2): 302-310.\\


\noindent
Mikal, J., \textbf{K.Grace}, J. DeWaard, M. Brown, G. Sangli. 2020. Domestic migration and mobile phones:  A qualitative case study focused on recent migrants to Ouagadougou, Burkina Faso. \textit{PLOS One}, 15 (8): e0236248.\\

\noindent
Randell*, H., C., Gray, \textbf{K. Grace}. 2020. Stunted from the Start: Early Life Weather Conditions and Child Undernutrition in Ethiopia. \textit{Social Science \& Medicine}, 261: 113234.\\

\noindent
Verdin, A., Funk, C., Peterson, P., Landsfeld, M., Tuholske*, C., \textbf{Grace, K.} 2020. Development and validation of the CHIRTS-daily quasi-global high-resolution daily temperature data set. 
\textit{Scientific Data}, 7(1): 1-14.\\

\noindent
Brown, M., T. Billing*, D. Backer, P. White, \textbf{K. Grace}, S. Doocy, P. Huth. 2020. Empirical Studies of Factors Associated with Child Malnutrition: Highlighting the Evidence about Climate and Conflict Shocks. \textit{Food Security}, 1-12.\\

\noindent
\textbf{Grace, K.}, F. Davenport, A. Dorelien. 2020.  Investigating the linkages between pregnancy outcomes and climate in sub-Saharan Africa. \textit{Population and Environment}, 1-25.\\

\noindent
\textbf{Grace, K.}, S. Billingsley, D. Van Riper.  2020. Building an Interdisciplinary framework to advance conceptual and technical aspects of population-environment research focused on women's and children's health. \textit{Social Science \& Medicine}, 112857.\\

\noindent
\textbf{Grace, K.}, A. Murray, R., Wei. 2020. Improving Urban and Peri-urban Health Outcomes Through Early Detection and Aid Planning.  In \textit{Global Perspectives on Health Geography: Geospatial Technologies for Urban Health}, pp. 231-250.\\

\noindent
Kugler, T., \textbf{K. Grace}, et al. 2019. People and Pixels 20 years later: The current data landscape and research trends blending population and environmental data. \textit{Population and Environment}, 41(2): 209-234.\\

\noindent
\textbf{Grace, K.}, N. Nagle, C. Burgert-Brucker, S. Rutzick*, D. Van Riper, T. Dontamsetti, T. Croft. 2019.  Integrating Environmental Context into DHS Analysis While Protecting Participant Confidentiality: A New Remote Sensing Method \textit{Population and Development Review}, 45(1), 197.\\

\noindent
\textbf{Grace, K.}, V., Hertrich, D. Singare*, G. Husak. 2018. Examining rural Sahelian out-migration in the context of climate change: An analysis of the linkages between rainfall and out-migration in two Malian villages from 1981 to 2009. \textit {World Development}, 109, 187-196.\\

\noindent
Bakhtsiyarava*, M., \textbf{K. Grace}, R. Nawrotzki. 2018. A Comparative Analysis of the Relationship Between Climate Change, Agricultural Specialization and Children's Health: Evidence from Kenya and Mali. \textit {American Journal of Public Health}, 108(S2), S144-S150.\\

\noindent
\textbf{Grace, K.} 2017. Considering climate in studies of fertility and reproductive health in poor countries. \textit{Nature Climate Change},  7: 479-485.\\

\noindent
Davenport, F.,  \textbf{K. Grace}, C. Funk, S. Shukla. 2017. Child Health Outcomes in Sub-Saharan Africa: A Comparison of Changes in Climate and Socio-economic Factors. \textit{Global Environmental Change}, 46, 72-87.\\

\noindent
\textbf{Grace, K.}, A. Lerner, J. Mikal, G. Sangli. 2017. A qualitative investigation of childbearing and seasonal hunger in peri-urban Ouagadougou, Burkina Faso.  \textit{Population and Environment}, 38(4): 369-380.\\

\noindent
\textbf{Grace, K.}, L. Frederick*, M. Brown, L. Boukerrou, and B. Lloyd*. 2017. Investigating Important Interactions between Water and Food Security for Child Health in Burkina Faso \textit{Population and Environment}, 39(1): 26-46.\\


\noindent
\textbf{Grace, K.}, R. Wei, and A. Murray 2017.  Using population, health and remotely sensed environmental data to improve food aid distribution in rural West Africa: An application of optimization methods. \textit{Food Security}, 9 (4), 867-880.\\


\noindent
Husak, G. and \textbf{K. Grace}. 2016. In search of a global model of cultivation:  Using remote sensing technology to examine the characteristics and constraints of agricultural production in the developing world \textit{Food Security}, 8(1): 167-177.\\

\noindent
\textbf{Grace, K.}, N. Nagle and G. Husak. 2016. Why are some vulnerable children healthy and others stunted?  A case study of vulnerability and resilience among very young children in Mali, West Africa \textit{Annals of the Association of American Geographers}, 35: 125-137.\\

\noindent
\textbf{Grace, K}. and S. Sweeney. 2016.  Decomposing Guatemala's fertility stall. \textit{Demography}, 53(1): 117-137.\\

\noindent
\textbf{Grace, K.}, F. Davenport*, H. Hanson, C. Funk, S. Shukla*. 2015. Examining the Relationship between Temperature, Rainfall and Low Birth Weight in sub-Saharan Africa.  \textit{Global Environmental Change}, 35:125-137.\\

\noindent
Brown, M.E., J.M. Antle, P. Backlund, E.R. Carr, W.E. Easterling, M.K. Walsh, C. Ammann, W. Attavanich, C.B. Barrett, M.F. Bellemare, V. Dancheck, C. Funk, \textbf{K. Grace}, J.S.I. Ingram, H. Jiang, H. Maletta, T. Mata, A. Murray, M. Ngugi, D. Ojima, B. O'Neill, and C. Tebaldi. 2015. Climate Change, Global Food Security, and the U.S. Food System. 146 pages.
 \\

\noindent
\textbf{Grace, K}. and N. Nagle. 2015. Using High Resolution Remotely Sensed Data to Examine the Relationship between Agriculture and Fertility in Mali.  \textit{The Professional Geographer}, 67(4): 641-654.\\

\noindent
\textbf{Grace, K}., G. Husak and S. Bogle*. 2014.  Estimating Agricultural Production in Marginal and Food Insecure Areas in Kenya Using Very High Resolution Remotely Sensed Imagery. \textit{Applied Geography}, 55:257-265.\\

\noindent
\textbf{Grace, K}., M. Brown and A. McNally*. 2014. Examining the link between food price and food insecurity:  A multi-level analysis of maize price and birthweight in Kenya.   \textit{Food Policy}, 46: 56-65. \\


\noindent
Brown, M., \textbf{K. Grace}, K.  G. Shively, K. Johnson, M. Carroll. 2014. Using Satellite Remote Sensing and Household Survey Data to Assess Human Health and Nutrition Response to Environmental Change. \textit{Population and Environment}, 1-25.\\


\noindent
\textbf{Grace, K}. and S. Sweeney. 2014.  Pathways to marriage and cohabitation in Central America.  \textit{Demographic Research}, 30(6): 187-226.\\

\noindent
Sweeney, S., F. Davenport* and \textbf{K. Grace}. 2013. Combining insights from quantile and ordinal regression: Child malnutrition in Guatemala  \textit{Economics and Human Biology}, 11(2):164-177.\\

\noindent
\textbf{Grace, K}. and S. Sweeney. 2013. Understanding stalling demographic transition in high-fertility countries: A case study of Guatemala.  \textit{Journal of Population Research}, 30:19-37. \\

\noindent
\textbf{Grace, K}, F. Davenport*, C. Funk, A. Lerner*. 2012. Child malnutrition and climate conditions in Kenya. \textit{Applied Geography}, 35(1):405-413. \\


\noindent
\textbf{Grace, K}., G. Husak, L. Harrison*, D. Pedreros, J. Michaelsen. 2012. Using high resolution satellite imagery to estimate cropped area in Haiti and Guatemala. \textit{Applied Geography} 32:433-440.\\


\noindent
Mikal, J. and \textbf{K. Grace}. 2012.  Against abstinence-only education abroad: Viewing internet use during study abroad as a possible experience enhancement. \textit{Journal of Studies in International Education} 16: 287-306.\\

\noindent
\textbf{Grace, K}. 2010. Contraceptive use and intent in Guatemala.  \textit{Demographic Research} 23: 335-364.\\




%\noindent
%\textbf{Submitted Papers (2)}


%\noindent
%Grace, K., F. Davenport and A. Dorelien ``The impact of climate and food insecurity on miscarriage, stillbirth and neonatal mortality in Africa" in submission\\


%\noindent
%Grace, K., V. Hertrich, D. Singare*, G. Husak ``Using high resolution rainfall data and detailed migration histories to examine climate vagaries and out-migration in rural Mali" revise and resubmit \\






%\vspace{.25cm}

%\noindent
%\textbf {Papers in Progress (3)}



%\noindent
%Grace, K.  ``Climate variability and health in extremely vulnerable communities: Investigating variations in surface water availability, vegetation and food insecurity in the Sahel" \\


%\noindent
%Grace, K, N. Nagle and C. Burgert-Brucker ``Seeking contextually relevant spatial information while protecting participant confidentiality:  An application of Tobler’s First Law"\\

%\noindent
%Grace, K, G. Husak, F. Davenport ``Merging fine and coarse resolution remotely sensed data with household-level survey data to evaluate small-scale vulnerability to climate change in West Africa"\\













\noindent
\textbf {Other Publications (5)}

\noindent
Balk, D., and Grace, K. (2019). Investigating demographic processes using innovative combinations of remotely sensed and demographic data. \textit{Population and Environment}, 1-3.\\

\noindent
Grace, K., and Mikal, J. (2019). Incorporating qualitative methodologies and fieldwork into large scale, quantitative analyses of climate-health in low-income countries. \textit{The Lancet Planetary Health}, 3(12), e496-e498.\\

\noindent
Brown, M. E., Carr, E. R., Grace, K. L., Wiebe, K., Funk, C. C., Attavanich, W., Backlund, P. and Buja, L. (2017). Do markets and trade help or hurt the global food system adapt to climate change? \textit{Food Policy}, 68, 154-159.\\

\noindent
Thenkabail, P.S., M.E. Brown, K.M. deBeurs, and K. Grace. (2015). Global Land Surface Phenology and Implications for Food Security. In \textit{Land Resources Monitoring, Modeling, and Mapping with Remote Sensing} (pp. 353-363). CRC Press. \\

\noindent
Sweeney, S. and K. Grace. 2012. Regression Analysis of Anthropometry
Data: A Simulation Study of a Two-Stage Estimator, In \textit{JSM
Proceedings}, Health Policy Statistics Section, Alexandria, VA:
American Statistical Association. \\

%\noindent
%Grace, K and D.L. Carr. 2009. Fertility in Pet\'en, Guatemala: The impact of the context, the household and the individual on fertility. \textit{MPIDR Working Paper} 2009-037, Max Planck Institute for Demographic Research, September.\\

%\noindent
%Grace, K. 2008. Guatemalan regional fertility patterns 1987-2002. \textit{Demographic and Health Survey Working Paper} WP51, Calverton, Maryland: Macro International. \\

\vspace{.5cm}

\noindent
{\large \textbf{PRESENTATIONS}}
\vspace{.5cm}

\noindent
\textbf {Invited Talks}


\noindent
Dorelien, A., and Grace, K.  `Data and the frameworks for linking women and children’s health to climate change,' Environmental Impacts on Families: Change, Challenge, and Adaptation at Penn State’s 29th Annual Symposium on Family Issues, Penn State University,  October 25-26, 2021, State College, PA.\\

\noindent
Grace, K. `Data, measurement, and pathways linking climate change and child health,' Demography Department, University of California, Berkeley, October 14, 2021, Berkeley, CA.\\

\noindent
Grace, K. `Climate change and fertility,' IUSSP, PERN, and EAPS joint webinar on Climate Change and Population Dynamics, September 16, 2021.\\

\noindent
Grace, K.  `Using spatial data to explore gender-based vulnerabilities in a context of climate change,'
UN sponsored seminar on Gender and Geospatial Analysis for the 2030 Agenda and the Sustainable Development Goals, El Colegio de Mexico, February 25, 2021, Mexico City, Mexico.\\

\noindent
Grace, K. ‘Climate variability and women and children’s health and well-being in vulnerable communities’, Humanitarian Data Science Division, University College London, November, 2020, London, England.\\

\noindent
Grace, K. `Heat, hunger, and resilience: Climate change and children's health', Population and Health Research Group, University of St. Andrews, September 29, 2020, Scotland, UK.\\

\noindent
Grace, K. `Heat, Hunger and Resilience: Climate and Children's Health', Department of Population Health, NYU Langone Health, April 28, 2020, NY, NY.\\

\noindent
Grace, K. `Considering climate and outmigration in two rural Malian villages', Workshop on Disentangling the Drivers of Migration in West Africa, Columbia University, February, 2020. NY, NY.\\

\noindent
Grace, K. `Exploring strategies for investigating the underlying linkages between climate and child health', Department of Earth and Planetary Sciences, Johns Hopkins University, February, 2020. Baltimore, MD.\\

\noindent
Grace, K. `Applying social science perspectives in the classroom to examine the impact of climate change  Department of Human Ecology, UC Davis, November, 2018. Davis, California.\\

%\noindent
%Grace, K. `Considering the linkages between climate and health: A focus on women'sâ and children'sâ health in sub-Saharan Africa'â Department of Human Ecology, UC Davis, November, 2018. Davis, California.\\

\noindent
Grace, K. `Considering the linkages between climate and health: A focus on women's and children's health in sub-Saharan Africa' CUNY Institute for Demographic Research (CIDR), November, 2018. NY, NY. \\


\noindent
Grace, K. `Considering the linkages between climate and health: A focus on women's and children's health in sub-Saharan Africa' Geography Department, Macalaster College, February, 2018. Saint Paul, MN. \\

\noindent
Grace, K. `Considering the linkages between climate and health: A focus on women's and children's health in sub-Saharan Africa' Carolina Population Center, University of North Carolina, Chapel Hill.  January, 2018. Chapel Hill, NC.\\

\noindent
Grace, K. 'Considering women and children's health in the context of climate change: an investigation of birthweight, temperature and precipitation in Africa' Population Studies and Training Center, Brown University.  November, 2017. \\ Providence, RI.\\

\noindent
Grace, K. `Investigating birthweight, temperature, and precipitation in Africa' Department of Geography, Clark University. September, 2017, Worcester, MA.\\

\noindent
Grace, K. `Food insecurity and Children's Health in Africa' Pediatric Grand Rounds, UMN Children's Hospital. September, 2016, Minneapolis, MN.\\

\noindent
Grace, K. `Agricultural Variability and the Impact on Children's Health' NASA Land Cover Land Use Change (LCLUC) Webinar.  May, 2016, USA.\\

\noindent
Grace, K. `Examining the link between climate change and birthweight in Africa' Famine Early Warning System (FEWS NET) Science Meeting.  April, 2016, Santa Barbara, CA.\\

\noindent
Grace, K. `Food insecurity and Children's Health in Africa' Institute for Population Research, The Ohio State University. February, 2016, Columbus, OH.\\

\noindent
Grace, K. and G. Husak `Agriculture and health in a context of climate change: Food availability under different climate scenarios in West Africa' NASA Land Cover Land Use Change (LCLUC) Meeting. April 2015, Washington DC.\\

\noindent
Grace, K. `Fertility, Food Insecurity and Agriculture in West Africa' Pennsylvania State University. May 2014, State College, PA.\\

\noindent
K. Grace, N.Nagle. `Using High Resolution Remotely Sensed Data to Re-Examine the Relationship between Agriculture and Fertility in a Pre-Transitional Setting'  Vienna Institute for Demography. May 2013, Vienna, Austria.\\

\noindent
Grace, K., and N. Nagle `Using High Resolution Imagery to Evaluate Fertility and Agriculture' October 2013,  Oak Ridge National Laboratory, TN. \\

\noindent
Grace, K., M. Brown and A. McNally `Maize Prices and Low Birth Weight in Kenya' October 2013,  University of Tennessee, Knoxville, TN. \\

\noindent
Grace, K., M. Brown and A. McNally 'Maize Prices and Low Birth Weight in Kenya' December 2012,  Human Health and Ecosystems Workshop at SESYNC in Annapolis, MD.\\

\noindent
Grace, K. `Union formation in Guatemala, Honduras and Nicaragua' March 2011, Auburn University, Auburn, AL. \\

\noindent
Grace, K. `Spatial analysis from a geographer's perspective', April 2010, Brown University, Providence, RI. \\

\noindent
Grace, K. `Using classification trees to analyze contraception in Guatemala', December 2008, University of Iowa, Iowa City, IA.\\

\noindent
Grace, K. `Using Trees to Reveal Better Fruit: Identifying high need populations in Guatemala', December 2008, University of Groningen, Holland.\\
 
%Carr, D.L., K. Grace, and J. Davis.  `Population and Land Cover Change in Latin America and 
%Guatemala. Symposium: The influence of human demography and agriculture on natural 
%systems in the Neotropics'  Association for Tropical Biology and Conservation Morelia, 
%Mexico July 15-19, 2007.

\vspace{.5cm}


\noindent
\textbf {Papers Accepted for Presentation at Academic Association Meetings }

\noindent





\noindent
Grace, K,. A. Verdin and V. Hertrich. `How Do Changing Environmental Factors Interact With Individual Factors to Influence Migration Behavior in Environmentally Precarious Communities'  American Association of Geographers (AAG) Annual Meeting April 3-7, 2019, Washington, DC.\\

\noindent
Grace, K,. S. Billingsley. `Fertility and Climate Instability in Agriculturally-Dependent Contexts: Parity Transitions in Albania, Moldova, and Uzbekistan ' Population Association of America (PAA) Annual Meeting April 10-13, 2019, Austin, TX.\\

\noindent
Grace, K,. A. Verdin and V. Hertrich `How Do Changing Environmental Factors Interact With Individual Factors to Influence Migration Behavior in Environmentally Precarious Communities: Investigating Outmigration From 1981-2009 in Rural Mali' Population Association of America (PAA) Annual Meeting April 10-13, 2019, Austin, TX.\\

\noindent 
Grace, K., A. Verdin, and V. Hertrich `How do changing environmental factors influence internal migration behavior in environmentally precarious communities in Mali, 1981-2008', Migration, Environment and Climate: What Risk Inequalities?, INED, October, 2018.  Paris, France. \\

\noindent
Randell, H., C. Gray, K.Grace. `Climate Change and Multidimensional Vulnerability to Child Undernutrition: Evidence from Ethiopia.' Population Association of America (PAA) Annual Meeting April 26-28, 2018, New Orleans, LA.\\

\noindent
K. Grace, F. Davenport, A. McNally, G. Husak. `Climate Variability and Health in Extremely Vulnerable Communities: Investigating Variations in Surface Water Availability, Vegetation, and Food Insecurity in the Sahel.' Population Association of America (PAA) Annual Meeting April 26-28, 2018, New Orleans, LA.\\

\noindent
A. Dorelien, K. Grace, F. Davenport. `Malaria Exposure and Pregnancy Outcomes in Sub-Saharan Africa.' Population Association of America (PAA) Annual Meeting April 26-28, 2018, New Orleans, LA.\\


\noindent
Grace, K., M. Brown, M. Bakhtsiyarava. 'A multi-scalar investigation of development and health in Ethiopia: Household electrification in an agriculturally dependent and climate sensitive country'. American Geophysical Union (AGU) Fall Meeting, December 11-15, 2017. New Orleans, LA.\\

\noindent
Bakhtisyarava, M., K. Grace, R. Nawrotzki. `Does Agricultural Specialization Matter? An Analysis of the Relationship between Climate Change, Agricultural Specialization, and Children€™s Health in Kenya and Mali' International Population conference at the International Union for the Scientific Study of Population (IUSSP). October 29-November 4, 2017, Cape Town, South Africa.\\

\noindent
Grace, K.,  F. Davenport, A. Dorelien `The Impact of Climate and Food Insecurity on Miscarriage,
Stillbirth and Neonatal Mortality in Africa' Population Association of America (PAA) Annual Meeting.
April 27-April 29, 2017, Chicago, IL.\\

\noindent
Grace, K., N. Nagle, C. Burger-B, S. Rutzick, D. Van Riper `Seeking Contextually Relevant Spatial
Information While Protecting Participant Confidentiality: An Application of Tobler's First Law'
Population Association of America (PAA) Annual Meeting. April 27-April 29, 2017. Chicago, IL.\\

\noindent
Bakhtsiyarava, M.,  K. Grace, R. Nawrotzki `Does Agricultural Specialization Matter? A Comparative
Analysis of the Relationship Between Climate Change, Agricultural Specialization, and Children's Health
in Kenya and Mali' Population Association of America (PAA) Annual Meeting. April 27-April 29, 2017
Chicago, IL.\\

\noindent
Grace, K., Davenport, F., Dorelien, A. `Investigating the Linkages between Pregnancy Outcomes and Climate in sub-Saharan Africa' American Association of Geographers Annual Meeting  (AAG).  April 5-7, 2017 Boston, MA. \\



\noindent
Davenport, F., K. Grace, C. Funk, S. Shukla. `Child health outcomes in Sub-Saharan Africa: A comparison of changes in climate and socio-economic factors' Western Regional Science Association (WRSA) Annual Meeting. February 15-18, 2017, Santa Fe NM.\\


\noindent
Grace, K.,  G. Husak, F. Davenport. `Merging fine and coarse resolution remotely sensed data with household-level survey data to evaluate small-scale vulnerability to climate change in West Africa'  American Geophysical Union (AGU) Fall Meeting.  December 12-16, 2016, San Francisco CA.\\ 

\noindent
Grace, K., V. Hertrich, D. Singare, G. Husak. `Examining Rural Sahelian out-Migration in the Context of Climate Change: A Multi-Method Analysis of the Linkages Between Rainfall and Outmigration in Two Rural Villages from 1981-2009' Population Association of America (PAA) Annual Meeting. March 31-April 2, 2016, Washington DC.\\

\noindent
Grace, K., R. Wei, A. Murray. `A Spatial Analytic Framework for Assessing and Improving Food Aid Distribution in Developing Countries' Western Regional Science Association (WRSA) Annual Meeting. February 14-17, 2016, Big Island HI.\\

\noindent
Grace, K., L. Frederick, M. Brown, L. Boukerrou, and B. Lloyd. `The Impacts of Water Quality and Food Availability on Children’s Health in West Africa: A Spatial Analysis Using Remotely Sensed Data and Small-Scale Water Quality Data and Country-level Health Data' American Geophysical Union (AGU) Fall Meeting.  December 14-18, 2015, San Francisco CA.\\

\noindent
K. Grace, N. Nagle and G. Husak. `Why are some vulnerable children healthy and others stunted? A case study of vulnerability and resilience among very young children in West Africa' American Geophysical Union (AGU) Fall Meeting.  December 15-19, 2014, San Francisco CA.\\

\noindent
K. Grace, F. Davenport, H. Hanson, C. Funk, S. Shukla. `Examining the Relationship between Temperature, Rainfall and Low Birth Weight: Evidence from 21 African Countries' European Population Conference (EPC).  June 25-28, 2014, Budapest, Hungary. \\

\noindent
K. Grace, F. Davenport, H. Hanson, C. Funk, S. Shukla. `Examining the Relationship between Temperature, Rainfall and Low Birth Weight: Evidence from 21 African Countries' Population Association of America (PAA) Annual Meeting.  May 2-4, 2014, Boston, MA. \\

\noindent
S. L. Lewis-Gonzales, N. Nagle, K. Grace. `Accuracy of supervised classification of cropland in sub-Saharan Africa’  American Geophysical Union (AGU) Fall Meeting. December 9-13, 2013, San Francisco, CA.\\

\noindent
K. Grace, A. Lerner, G. Sangli. `Women's Agency and Perception of Vulnerability: A Qualitative Analysis of Breastfeeding, Contraception and Food Insecurity in Burkina Faso' 2013 Annual Sociology of Development Conference. October 25-26, Salt Lake City, UT, USA.\\

\noindent
S. Sweeney and K. Grace. `Immigration and Fertility Change in Los Angeles' 2013 Annual Sociology of Development Conference. October 25-26, Salt Lake City, UT, USA.\\


\noindent
K. Grace, N. Nagle. `Using High Resolution Remotely Sensed Data to Re-Examine the Relationship between Agriculture and Fertility in a Pre-Transitional Setting'  Agricultural and Applied Economics Annual Meeting. August 4-6, 2013, Washington, DC, USA.\\

\noindent
K.Grace, N. Nagle. `Using High Resolution Remotely Sensed Data to Re-Examine the Relationship between Agriculture and Fertility in a Pre-Transitional Setting' Population Association of America (PAA) Annual Meeting.  April 11-13, 2013, New Orleans, LA, USA. \\

\noindent
K. Grace, M. Brown and A. McNally. `Maize Prices and Low Birth Weight in Kenya'  Population Association of America (PAA) Annual Meeting. May 3 - May 5, 2012, San Francisco, CA, USA.\\
\url{http://paa2012.princeton.edu/abstracts/122721}\\

\noindent
Sweeney, S. and K. Grace.  `Fertility Assimilation Behavior of Los Angeles Immigrants',
Population Association of America (PAA) Annual Meeting.  March 31 - April 2, 2011, Washington, DC, USA.\\
\url{http://paa2011.princeton.edu/abstracts/112304}\\

\noindent
Grace, K., F. Davenport and C. Funk.  `Fertility, Child Stunting, and Climate Change: An analysis of recent trends in Kenya',
Population Association of America (PAA) Annual Meeting.    March 31 - April 2, 2011, Washington, DC, USA.\\
\url{http://paa2011.princeton.edu/abstracts/111992}\\


\noindent
Grace, K and G. Husak. `Estimating Cropped Area in Haiti and Guatemala', European Population Conference. 1-5 September, 2010, Vienna, Austria.\\
\url{http://epc2010.princeton.edu/abstracts/100512}\\

\noindent
Grace, K and G. Husak. `Using Demographic Data to Estimate Cropped Area in Highly Undernourished Countries', Association of American Geographers Annual Conference. 14-18 April, 2010, Washington D.C., USA.\\

\noindent
Grace, K. and S. Sweeney.  `Informal and Formal Union Formation in Three Central American Countries',
Population Association of America (PAA) Annual Meeting.  April 28 -May 2, 2009, Detroit, USA.\\
\url{http://paa2009.princeton.edu/abstracts/90963}\\

\noindent
Grace, K. `A comparative analysis of contraceptive use and intent in Guatemala',
Association of American Geographers Annual Conference. 22-27 March, 2009, Las Vegas, USA.\\

%\noindent
%Grace, K. and D. L. Carr. `Fertility in Pet\'en, Guatemala: The impact of the context, the household and the individual on fertility behavior', Population 
%Association of America (PAA) Annual Meeting. 17-19 April 2008, New Orleans, USA. \\
%\url{http://paa2008.princeton.edu/abstracts/80692}\\

\noindent
Grace, K.  `Fertility Correlates in Pet\'en, Guatemala',
Association of American Geographers Annual Conference. 17-21 April, 2007, San Francisco, USA.\\

\vspace{.5cm}

\noindent
 \textbf {Posters Accepted for Presentation at Academic Association Meetings }
 
 \noindent
Verdin, A. K. Grace, C. Funk, F. Davenport, G. Husak. `Advancing climate-health research with stochastic weather generators', American Geophysical Union (AGU)
 Fall Meeting December 2019, San Francisco, CA.\\

\noindent
Bakhtsiyarava, M. and K.Grace. `Investigation of the Determinants of Food Security: The Role of Agricultural Inputs for Household Food Security and Child Nutrition in Ethiopia.' Population Association of America (PAA) Annual Meeting April 26-28, 2018, New Orleans, LA.\\

\noindent
Grace, K., F. Davenport, C. Funk. `The Future of Infant Health and Mortality in Sub-Saharan Africa: Evaluating the Relative Importance of Changes in Socio-Economics versus Climate' Population 
Association of America (PAA) Annual Meeting. 30 April-2 May 2015, San Diego, USA. \\

\noindent
Sweeney, S. and K. Grace. `Immigrant fertility adaptation and change in Los Angeles', European Population Conference (EPC). June 25-28, 2014, Budapest, Hungary.\\

\noindent
Grace, K.,  A. Lerner, G. Sangli. `Breastfeeding, Contraception and Food Insecurity in Burkina Faso', Population Association of America (PAA) Annual Meeting.  May 2-4, 2014, Boston, MA, USA. \\

\noindent
Grace, K., G. Husak and S. Bogle, `Using Very High Resolution Remotely Sensed Imagery to Estimate Agricultural Production in Two Kenyan Regions: A comparison of marginal and high productive agricultural zones ', American Geophysical Conference (AGU), December 9-13, 2013, San Francisco, CA, USA.\\

 
 \noindent
Sweeney, S. and K. Grace, `Assessing immigrant fertility adaptation: Recurrent event models of birth and migration histories', SAM 2012 workshop: The statistical analysis of multi-outcome data. Paris, France.\\

\noindent
Grace, K. and S. Sweeney, `Examining Parity Transitions During Fertility Stall' European Population Conference (EPC). 12-14 June 2012, Stockholm, Sweden.\\

\noindent
Grace, K.,  J. Goldstein and Y.-H. Cheng `The Susceptibility of Fertility Measures to Random Fluctuations', European Population Conference (EPC). 1-5 September 2010, Vienna, Austria. \\

\noindent
Grace, K. and S. Sweeney. `Melting pot or mixed salad? Exploring assimilation and 
maintenance of fertility behavior of first and subsequent generation Central American 
women living in the United States', International Union of the Scientific Study of 
Population (IUSSP). September 27 - October 2, 2009, Marrakech, Morocco. \\
\url{http://iussp2009.princeton.edu/abstracts/92077}\\


\noindent
Grace, K. `Contraceptive use and intent in Guatemala', Population Association of 
America (PAA) Annual Meeting. April 28 - May 2, 2009, Detroit, USA.\\
\url{http://paa2009.princeton.edu/abstracts/90132}\\


%\noindent
%Grace, K., S. Sweeney, and D. L. Carr  `A Spatial and Temporal Analysis of Fertility in Guatemala', Population 
%Association of America (PAA) Annual Meeting. 17-19 April 2008, New Orleans, USA. \\
%\url{http://paa2008.princeton.edu/abstracts/81122}\\

\noindent
{\large \textbf{TEACHING AND GRADUATE MENTORSHIP}}
\vspace{.5cm}

%\begin{tabular}{lcp{8cm}}
\noindent
\textbf{Postdoctoral researchers and research scientists}:\\
Andrew Verdin (2018-2020)\\
Nina Brooks (2020-2021)\\
Jacqueline Banks (postdoctoral trainee MPC) (2019-2021)\\
Maya Luetke (2021-ccurrent)\\
Jiao Yu (2021-current)\\


\noindent
\textbf{Committee Chair (Ph.D.)}: \\
Maryia Bakhtsiyarava (co-chair) (completed PhD 2020, beginning postdoctoral fellowship at UC Berkeley, fall 2020)\\
Ruthie Burrows\\
Grace Cooper\\
Oforiwaa Pee Agyei-Boakye (co-chair)\\

\noindent
\textbf{Committee Membership (Ph.D.)}: \\
Marin Bryce (completed PhD 2018)\\
Han Li (completed PhD 2016)\\
Jacqueline Banks (completed PhD 2019)\\
Kwame Tsikudo (completed PhD 2019)\\
Brent Lloyd (completed PhD 2019)\\
Aaron Mallory (completed PhD 2020)\\
Jacqueline Daigneault (completed PhD 2020) \\
Anna Bolgrien (completed PhD 2021)\\
Morrison (Luke) Smith (in progress) \\
Jasper Johnson (in progress)\\
Isaac Asante-Wusu (in progress) \\

\noindent
\textbf{Committee Chair (Masters)}: \\
Julia Reich (advised until Fall 2018)\\
 Lori Miles (2014) \\
 Delanie Farnham (2015)\\
 Seth Bogle (advised until Fall 2016) \\
 Michael Ryba (advised until Fall 2015)\\


\noindent
\textbf{Committee Membership (Masters)}: \\
Kaila McDonald (2014)\\
Emily Nicolossi (2015) \\
Jessie Bakker (Completed MA, 2020)\\





\noindent
\textbf{Undergraduate Mentorship}:
 Maximilian Stiefel (honors thesis 2014, currently PhD student at UC Santa Barbara Geography)\\
 Marc Healy (Undergraduate Research Opportunity Program grant recipient, 2013/14)\\
 Eddy Urena (Undergraduate Research Opportunity Program grant recipient, 2018)\\
 Suzanne Scotty (Undergraduate Research Opportunity Program grant recipient, 2020)\\
 Kassandra Fate (Capstone Project)\\

\vspace{0.5cm}
\noindent
\textbf{Teaching Experience}

\noindent
Geography of Health and Health care, University of Minnesota, Department of Geography, Environment and Society, Spring 2016, Spring 2017, Fall 2017, Fall 2018, Fall 2019, Fall 2020\\
Global Health Survey Data analysis Spring 2019, Fall 2020, Spring 2021\\
Population Geography, University of Minnesota, Department of Geography, Environment and Society, Spring 2017, Fall 2018\\
Graduate Seminar: Statistical Methods, University of Minnesota, Department of Geography, Environment and Society, Fall 2017\\
Analytic Methods for Population Research, University of Utah, Department of Geography, Fall 2013\\
Economic Geography, University of Utah, School of Business, Spring 2013, Spring 2014, Fall 2014, Fall 2015\\
Global Development, University of Utah, Department of Geography, Spring 2014\\
Advanced Population Geography, University of Utah, Department of Geography, Spring 2013\\
Population Geography, University of Utah, Department of Geography, Fall 2012, 2013, 2014, Fall 2015\\
Guest Lecturer, Malnutrition and climate in Kenya in Spatial Statistics, Spring 2011\\
Guest Lecturer, Fertility Analysis in Spatial Demography, Winter 2010 and Spring 2012 \\
Teaching Assistant, Center for Spatially Integrated Social Sciences (CSISS) -  Spatial Perspectives on Analysis for Curriculum Enhancement (SPACE), UCSB, 2005 - 2007\\
Graduate Research Mentor (high school summer outreach program), UCSB, 2005 - 2007\\
Teaching Assistant, Human Geography, UCSB, Summer 2007\\
Guest Lecturer, Global trends in Fertility for Population Geography, Winter 2007\\
Teaching Assistant, Center for Spatially Integrated Social Sciences (CSISS) -  Population Science Summer Workshop, UCSB, 2005 - 2006\\
Teaching Assistant, Population Geography, UCSB, Fall 2005\\
Teaching Assistant, Introductory Statistics, UCSB, 2004 - 2005\\





\vspace{.5cm}
\noindent
{\large \textbf{PROFESSIONAL SERVICE}}
\vspace{.5cm}

\noindent

\noindent
\textbf{Discipline}\\
Editorial Board Discover Social Science and Health Journal 2021-\\
Editorial Board Computers, Environment and Urban Systems Journal 2020-\\
Editorial Board CABI Agriculture and Biosciences Journal 2021-\\
Editorial Board Spatial Demography 2018-\\
Editorial Board Population and Environment 2018-\\
Chair, Population and Environment Research Network 2021-\\
President, Population Specialty Group American Association of Geographers 2019-2021\\
Co-guest editor, Frontiers - Gender and social consideration in climates and impacts research 2020-\\
Advisory Board Population Environment Research Network 2019-2021\\
Organizing Committee Population Association of America Annual Meeting 2020\\
Session Organizer - Population Association of America Annual Meeting 2019 \\
Session Chair - Population Association of America Annual Meeting 2018\\
Discussant - Population Association of America Annual Meeting 2018\\
Guest Editor Population and Environment - Special Issue 2018-2019\\
Vice-President, Population Specialty Group American Association of Geographers 2017-2019\\
Guest Editor Population and Environment 2018 and 2020\\
Discussant - Population Association of America Annual Meeting 2015\\
Student Poster Judge - American Geophysical Union Annual Meeting 2014\\
Session Co-Organizer - Population Association of America Annual Meeting 2014 \\


\noindent
\textbf{University of Minnesota}\\
Diversity, equity and inclusion Committee Chair, Department of GES (2020-2021)\\
MPC Senior Faculty Search Committee, Minnesota Population Center (2018-2019)\\ 
GIS Faculty Search Committee co-Chair, Department of GES (2019)\\
Ad-hoc Merit Review Committee, Department of GES (2018-2019)\\
Merit Committee Chair, Department of GES (2018-2019)\\
Grievance Committee Chair, Department of GES (2018-2019 & 2020-2021)\\
Coffee Hour/Brown Day Organizing Committee Chair - Department of GES (2017-2018)\\
Graduate Education Policy Committee - Department of GES (2017-2020)\\
Admissions Committee - Department of GES (2016-2017)\\
Executive Committee -  Department of GES (2016-2018)\\ 
Advisory Board Member - DASH (2018-current)\\
Fulbright Grant Reviewer (UMN) - (2017-2018)\\
Graduate Travel Grant Reviewer - (2017-2019)\\
Graduate School Advisory Board Member (2016-2019)\\
Advisory Board Member - Minnesota Population Center (2016-2019)\\
Co-Organizer (with Miranda Joseph) of Counter Accounting Workshop (2017)\\
Grants and Awards committee - Department of GES 2016\\
Parenting Across Roles, Invited Panelist Fall 2016\\
Parenting in Academia, Invited Panelist Spring 2016\\

\noindent
\textbf{University of Utah}\\
Moderator - University of Utah, Undergraduate Research Conference 2014\\
College of Social and Behavioral Sciences, Appeals Committee 2013 - 2015\\
Department of Geography, Graduate Committee 2013 - 2015\\ 
Department of Geography, Colloquia Committee 2012 - 2015\\




\noindent
\textbf{Professional Membership}

\noindent
Association of American Geographers\\
American Geophysical Union\\
Population Association of America\\
American Statistical Association\\

\vspace{.5 cm}
\noindent
\textbf {Reviewer}

\noindent 
Nature Energy\\
Nature Sustainability (x2)\\
Nature Climate Change\\
Nature Communication\\
Proceedings of the National Academy of Sciences\\
Science\\ 
Lancet Planetary Health (x2)\\
Demography (x4)\\
American Sociological Review\\
American Journal of Public Health\\
Population and Development Review (x2)\\ 
Scientific Data\\
Food Policy\\
Annals of the American Association of Geographers\\
American Journal of Agricultural Economics\\
Global Environmental Change (x3)\\
Global Change Biology\\
Population and Environment (x6)\\
Spatial Demography (x2)\\
Population (French Language) (x2)\\
International Regional Science Review\\
Scientific Reports\\
Health and Place\\
BMC Nutrition\\ 
Public Health Nutrition\\
Social Science and Medicine\\
International Perspectives on Sexual and Reproductive Health (x2)\\ 
PLOS One (x3)\\
Remote Sensing Applications: Society and Environment\\
Health Science Journal\\
Health and Place\\
International Journal of Disaster Risk Reduction\\
Landscape Ecology\\
Health Education Journal\\
Applied Geography (x4)\\ 
Pan American Journal of Public Health (x3)\\ 
Remote Sensing\\
Urban Studies\\
Population Review\\


\noindent
NASA LCLUC ROSES\\
%NIH review panel (CHHD-W), 2021-\\
Swiss National Science Foundation (x2)\\
NASA Postdoctoral Program\\
National Science Foundation\\
EPA Star Graduate Student Fellowship\\
U of U Air Quality Program Seed Grant Funding Opportunity\\
NASA SERVIR Applied Sciences Grant Opportunity\\
JPI (Joint Programming Initiative) ``Connecting Climate Knowledge for Europe"\\






\vspace{.5cm}

\noindent
{\large \textbf{COMMUNITY SERVICE}}
\vspace{.5cm}


\noindent
Santa Barbara Charter School Parent Volunteer, 2011-2012\\
Goleta Family School Parent Volunteer and Goleta Earth Day co-Organizer, 2006 - 2008\\
Tulane University School of Public Health Student Health Representative, 2003 - 2004\\
UC Berkeley Student Parent Organization Treasurer, 2001 - 2002\\
UC Berkeley Family Student Housing Health Worker, 2001 - 2002\\

\vspace{.5cm}

\noindent
\textbf {Languages and Skills}

\noindent
French, 
SAS, SPSS, R, ArcGIS, \LaTeX\,  

\end{document}